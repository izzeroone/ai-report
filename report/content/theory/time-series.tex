\textbf{Khái niệm} \\
Mặt lượng của hiện tượng thường xuyên biến động qua thời gian. Trong thống kê để nghiên cứu sự biến động này ta thường dựa vào dãy số thời gian.
Dãy số thời gian là dãy số các trị số của chỉ tiêu thống kê được sắp xếp theo thứ tự thời gian.
Ví dụ: có số liệu về doanh thu của Bưu điện X từ năm 1999 - 2003 như sau: ĐVT: tỷ đồng. \\
\begin{table}[H]
	\begin{tabularx}{\textwidth}{X | X | X | X | X | X } 
		%\hline
		Năm		& 1999  & 2000  & 2001  & 2002  & 2003  		\\ \hline
		Doanh thu   & 23,9 & 28,1   & 37,3  & 47,2   	&67,4	   %\hline
	\end{tabularx}
	\label{tab:table1}
	\caption{}
\end{table}


Ví dụ trên đây là một dãy số thời gian về chỉ tiêu doanh thu của đơn vị Bưu điện này từ năm 1999- 2003. Qua dãy số thời gian có thể nghiên cứu các đặc điểm về sự biến động của hiện tượng, vạch rõ xu hướng và tính quy luật của sự phát triển, đồng thời để dự đoán các mức độ của hiện tượng trong tương lai.
Mỗi dãy số thời gian có hai thành phần:
\begin{itemize}
    \item Thời gian: có thể là ngày, tuần, tháng, quí, năm, . . . . Độ dài giữa hai thời gian liền nhau được gọi là khoảng cách thời gian.
    \item Chỉ tiêu về hiện tượng nghiên cứu: chỉ tiêu này có thể là số tuyệt đối, số tương đối, số bình quân. Trị số của chỉ tiêu còn gọi là mức độ của dãy số.
\end{itemize}

\textbf{Phân loại dãy số thời gian:} \\
Căn cứ vào tính chất thời gian của dãy số, có thể phân biệt thành 2 loại:
\begin{enumerate}
    \item Dãy số thời kỳ: là dãy số biểu hiện mặt lượng của hiện tượng qua từng thời kỳ nhất định
    \item Dãy số thời điểm: là loại dãy số biểu hiện mặt lượng của hiện tượng qua các thời điểm nhất định. Dãy số này còn được phân biệt thành 2 loại:
    \begin{itemize}
        \item Dãy số thời điểm có khoảng cách thời gian đều nhau.
        \item Dãy số thời điểm có khoảng cách thời gian không đều.
    \end{itemize}
\end{enumerate}


\textbf{Các yếu tố ảnh hưởng đến biến động thời gian:} \\
\begin{enumerate}
    \item Biến động có xu hướng.
    \item Biến động theo thời vụ.
    \item Biến động theo chu kỳ.
    \item Biến động bất thường.
\end{enumerate}

