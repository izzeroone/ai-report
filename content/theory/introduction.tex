\section{Khái niệm}
\label{sec:intro:khainiem}
Dự báo đã hình thành từ đầu những năm 60 của thế kỉ 20. Khoa học dự báo với tư cách một ngành khoa học độc lập có hệ thống lí luận, phương pháp luận và phương pháp hệ riêng nhằm nâng cao tính hiệu quả của dự báo. Người ta thường nhấn mạnh rằng một phương pháp tiếp cận hiệu quả đối với dự báo là phần quan trọng trong hoạch định. Khi các nhà quản trị lên kế hoạch, trong hiện tại họ xác định hướng tương lai cho các hoạt động mà họ sẽ thực hiện. Bước đầu tiên trong hoạch định là dự báo hay là ước lượng nhu cầu tương lai cho sản phẩm hoặc dịch vụ và các nguồn lực cần thiết để sản xuất sản phẩm hoặc dịch vụ đó.
Như vậy, dự báo là một khoa học và nghệ thuật tiên đoán những sự việc sẽ xảy ra trong tương lai, trên cơ sở phân tích khoa học về các dữ liệu đã thu thập được.
Khi tiến hành dự báo ta căn cứ vào việc thu thập xử lý số liệu trong quá khứ và hiện tại để xác định xu hướng vận động của các hiện tượng trong tương lai nhờ vào một số mô hình toán học.
Dự báo có thể là một dự đoán chủ quan hoặc trực giác về tương lai. Nhưng để cho dự báo được chính xác hơn, người ta cố loại trừ những tính chủ quan của người dự báo.
Ngày nay, dự báo là một nhu cầu không thể thiếu được của mọi hoạt động kinh tế - xác hội, khoa học - kỹ thuật, được tất cả các ngành khoa học quan tâm nghiên cứu.
\section{Ý nghĩa}
\label{sec:intro:ynghia}
\begin{itemize}
    \item Dùng để dự báo các mức độ tương lai của hiện tượng, qua đó giúp các nhà quản trị doanh nghiệp chủ động trong việc đề ra các kế hoạch và các quyết định cần thiết phục vụ cho quá trình sản xuất kinh doanh, đầu tư, quảng bá, quy mô sản xuất, kênh phân phối sản phẩm, nguồn cung cấp tài chính... và chuẩn bị đầy đủ điều kiện cơ sở vật chất, kỹ thuật cho sự phát triển trong thời gian tới (kế hoạch cung cấp các yếu tố đầu vào như: lao động, nguyên vật liệu, tư liệu lao động... cũng như các yếu tố đầu ra dưới dạng sản phẩm vật chất và dịch vụ). \\
    \item Trong các doanh nghiệp nếu công tác dự báo được thực hiện một cách nghiêm túc còn tạo điều kiện nâng cao khả năng cạnh tranh trên thị trường.\\
    \item Dự báo chính xác sẽ giảm bớt mức độ rủi ro cho doanh nghiệp nói riêng và toàn bộ nền kinh tế nói chung. \\
    \item Dự báo chính xác là căn cứ để các nhà hoạch định các chính sách phát triển kinh tế văn hoá xã hội trong toàn bộ nền kinh tế quốc dân \\
    \item Nhờ có dự báo các chính sách kinh tế, các kế hoạch và chương trình phát triển kinh tế được xây dựng có cơ sở khoa học và mang lại hiệu quả kinh tế cao.\\
    \item Nhờ có dự báo thường xuyên và kịp thời, các nhà quản trị doanh nghiệp có khả năng kịp thời đưa ra những biện pháp điều chỉnh các hoạt động kinh tế của đơn vị mình nhằm thu được hiệu quả sản xuất kinh doanh cao nhất. \\
\end{itemize}
\section{Vai trò}
\label{sec:intro:vaitro}
\begin{itemize}
    \item Dự báo tạo ra lợi thế cạnh tranh \\
    \item Công tác dự báo là một bộ phận không thể thiếu trong hoạt động của các doanh nghiệp, trong từng phòng ban như: phòng Kinh doanh hoặc Marketing, phòng Sản xuất hoặc phòng Nhân sự, phòng Kế toán – tài chính. \\
\end{itemize}
\section{Các loại dự báo}
\label{sec:intro:phanloai}
Căn cứ vào độ dài thời gian dự báo: Dự báo có thể phân thành ba loại
\begin{itemize}
    \item Dự báo dài hạn: Là những dự báo có thời gian dự báo từ 5 năm trở lên. Thường dùng để dự báo những mục tiêu, chiến lược về kinh tế chính trị, khoa học kỹ thuật trong thời gian dài ở tầm vĩ mô. \\
    \item Dự báo trung hạn: Là những dự báo có thời gian dự báo từ 3 đến 5 năm. Thường phục vụ cho việc xây dựng những kế hoạch trung hạn về kinh tế văn hoá xã hội... ở tầm vi mô và vĩ mô. \\
    \item Dự báo ngắn hạn: Là những dự báo có thời gian dự báo dưới 3 năm, loại dự báo này thường dùng để dự báo hoặc lập các kế hoạch kinh tế, văn hoá, xã hội chủ yếu ở tầm vi mô và vĩ mô trong khoảng thời gian ngắn nhằm phục vụ cho công tác chỉ đạo kịp thời. \\
\end{itemize}
Cách phân loại này chỉ mang tính tương đối tuỳ thuộc vào từng loại hiện tượng để quy định khoảng cách thời gian cho phù hợp với loại hiện tượng đó: ví dụ trong dự báo kinh tế, dự báo dài hạn là những dự báo có tầm dự báo trên 5 năm, nhưng trong dự báo thời tiết, khí tượng học chỉ là một tuần. Thang thời gian đối với dự báo kinh tế dài hơn nhiều so với thang
thời gian dự báo thời tiết. Vì vậy, thang thời gian có thể đo bằng những đơn vị thích hợp ( ví dụ: quý, năm đối với dự báo kinh tế và ngày đối với dự báo dự báo thời tiết).


Dựa vào các phương pháp dự báo: Dự báo có thể chia thành 3 nhóm
\begin{itemize}
    \item Dự báo bằng phương pháp chuyên gia: Loại dự báo này được tiến hành trên cơ sở tổng hợp, xử lý ý kiến của các chuyên gia thông thạo với hiện tượng được nghiên cứu, từ đó có phương pháp xử lý thích hợp đề ra các dự đoán, các dự đoán này được cân nhắc và đánh giá chủ quan từ các chuyên gia. Phương pháp này có ưu thế trong trường hợp dự đoán những hiện tượng hay quá trình bao quát rộng, phức tạp, chịu sự chi phối của khoa học - kỹ thuật, sự thay đổi của môi trường, thời tiết, chiến tranh trong khoảng thời gian dài. Một cải tiến của phương pháp Delphi – là phương pháp dự báo dựa trên cơ sở sử dụng một tập hợp những đánh giá của một nhóm chuyên gia. Mỗi chuyên gia được hỏi ý kiến và rồi dự báo của họ được trình bày dưới dạng thống kê tóm tắt. Việc trình bày những ý kiến này được thực hiện một cách gián tiếp ( không có sự tiếp xúc trực tiếp) để tránh những sự tương tác trong nhóm nhỏ qua đó tạo nên những sai lệch nhất định trong kết quả dư báo. Sau đó người ta yêu cầu các chuyên gia duyệt xét lại những dự báo của họ trên xơ sở tóm tắt tất cả các dự báo có thể có những bổ sung thêm. \\
    \item Dự báo theo phương trình hồi quy: Theo phương pháp này, mức độ cần dự báo phải được xây dựng trên cơ sở xây dựng mô hình hồi quy, mô hình này được xây dựng phù hợp với đặc điểm và xu thế phát triển của hiện tượng nghiên cứu. Để xây dựng mô hình hồi quy, đòi hỏi phải có tài liệu về hiện tượng cần dự báo và các hiện tượng có liên quan. Loại dự báo này thường được sử dụng để dự báo trung hạn và dài hạn ở tầm vĩ mô.
    \item Dự báo dựa vào dãy số thời gian: Là dựa trên cơ sở dãy số thời gian phản ánh sự biến động của hiện tượng ở những thời gian đã qua để xác định mức độ của hiện tượng trong tương lai.
\end{itemize}
