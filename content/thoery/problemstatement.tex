\textbf{Các phương pháp sử dụng}
2.1. Dự báo từ các mức độ bình quân.......................................................................28 2.1.1. Dự báo từ số bình quân trượt (di động).......................................................28 2.1.2. Mô hình dự báo dựa vào lượng tăng (giảm) tuyệt đối bình quân................29 2.1.3. Mô hình dự báo dựa vào tốc độ phát triển bình quân..................................30 2.2. Mô hình dự báo theo phương trình hồi quy (dự báo dựa vào xu thế)................32 2.2.1. Mô hình hồi quy theo thời gian ..................................................................33 2.2.2. Mô hình hồi quy giữa các tiêu thức.............................................................36 2.3. Dự báo dựa vào hàm xu thế và biến động thời vụ.............................................36 2.3.1. Dự báo vào mô hình cộng...........................................................................37 2.3.2. Dự báo dựa vào mô hình nhân....................................................................38 2.4. Dự báo theo phương pháp san bằng mũ............................................................42 2.4.1. Mô hình đơn giản ( phương pháp san bằng mũ đơn giản)..........................42
2.4.2. Mô hình xu thế tuyến tính và không có biến động thời vụ ( Mô hình san mũ
Holt – Winters)......................................................................................................46
2.4.3. Mô hình xu thế tuyến tính và biến động thời vụ..