Lê Viết Quốc khiến bạn không khỏi bất ngờ khi anh chính là một nhân vật quan trọng trong lĩnh vực trí tuệ nhân tạo tại Google. Quốc được biết đến với "Google Brain".

Vào năm 2014, Quốc đề xuất trình tự chuỗi (Seq2seq) học với nhà nghiên cứu Google Ilya Sutskever và Oriol Vinyals. Nó là một khung công cụ - một thư viện các mã lệnh (framework) giải mã bộ mã hóa có mục đích đào tạo các mô hình để chuyển đổi các chuỗi từ một tên miền này sang miền khác, chẳng hạn như chuyển đổi các câu sang các ngôn ngữ khác nhau.

Seq2seq learning đòi hỏi ít sự lựa chọn trong thiết kế kỹ thuật hơn và cho phép hệ thống dịch của Google hoạt động hiệu quả và chính xác trên các tệp dữ liệu khổng lồ. Nó chủ yếu được sử dụng cho các hệ thống dịch máy và được chứng minh là có thể ứng dụng được ở nhiều mảng hơn, bao gồm tóm tắt văn bản, các cuộc hội thoại với trí tuệ nhân tạo, và trả lời câu hỏi.

Sau đó, Quốc tiếp tục phát minh ra Doc2vec – một thuật toán không giám sát sử dụng cho việc hiển thị các nội dung có độ dài cố định từ các đoạn văn bản có độ dài biến đổi, chẳng hạn như câu, đoạn văn và các tài liệu.

Doc2vec là phần mở rộng của Word2vec, được giới thiệu vào năm 2013 bởi nghiên cứu sinh của Google, Tomas Mikolov. Ý tưởng của nó là mỗi từ có thể được biểu diễn bằng một vec-tơ, có thể được tự động học từ một tập hợp văn bản. Quốc sử dụng vector cho các đoạn văn để mô hình có thể tạo ra sự hiển thị - trình chiếu của tài liệu, bất chấp độ dài của nó.

Những nỗ lực nghiên cứu của Quốc đã được đền đáp. Trong năm 2016, Google đã công bố hệ thống dịch máy Nơ-ron (Neural Machine Translation System), sử dụng trí tuệ nhân tạo AI để tạo ra các bản dịch tốt hơn và tự nhiên hơn.