Công nghệ truyền thông đang tiến bộ rất nhanh. Và ngày nay, hầu hết mọi người thích kết nối thông qua tin nhắn văn bản hoặc ứng dụng nhắn tin.
Để làm cho cuộc sống thuận tiện hơn, các tổ chức và ngành công nghiệp đang thúc đẩy truyền thông bằng cách phát triển và đầu tư vào chatbot.

Chatbots là các dịch vụ được lập trình trên máy tính có thể tương tác như con người thông qua giao diện trò chuyện, cả về mặt văn bản và thính giác. Còn được gọi là talkbots, smartbots, bot, chatterbots hoặc tác nhân tương tác, các chương trình này nhằm giao tiếp với người dùng và hành xử như thể một con người thực sự đứng sau cuộc trò chuyện.

Hiện tại, các bot này được tìm thấy trong các giải pháp trò chuyện hoặc nhắn tin lớn như Facebook Messenger, Kik, Slack, WeChat, Line, LiveChat và Telegram. Với sự phổ biến của công nghệ, ngay cả những người hầu như không chú ý đến công nghệ hiện đại có lẽ đã tương tác với các bot này nhiều lần rồi.

\textbf{Hai loại Chatbots} \\[0.2em]
Chatbots không phải là mới. Tuy nhiên, việc sử dụng các bot đã thu hút các ngành công nghiệp trong vài năm qua. Được thành lập lần đầu tiên vào những năm 1960, chatbot đã đi một chặng đường dài từ sự phát triển ban đầu của nó. Có hai loại chatbot. Loại chatbot phổ biến nhất là dựa trên quy tắc và loại tiên tiến hơn được cung cấp trí tuệ nhân tạo.
Các chatbot trí tuệ nhân tạo (AI) sử dụng các hệ thống xử lý ngôn ngữ tự nhiên. Trong hệ thống này, các máy tính được lập trình để đọc, xử lý và phân tích số lượng lớn dữ liệu ngôn ngữ tự nhiên. Các công nghệ trí tuệ nhân tạo cũng bao gồm các thuật toán học sâu và máy học. Các bot AI học hỏi từ các cuộc trò chuyện và tương tác họ có với mọi người, mở rộng cơ sở dữ liệu của họ.
Mặt khác, các bot dựa trên quy tắc được tạo thành từ các hệ thống đơn giản và do đó, có các phản hồi hạn chế. Hệ thống quét và xác định từ khóa hình thành đầu vào của người dùng và trả lời bằng lệnh tương ứng. Không giống như các chatbot dựa trên AI, các chatbot dựa trên quy tắc không còn phản hồi khi chúng gặp các lệnh lạ và các từ không được nhận dạng.

\textbf{Sáng tạo công nghệ} \\[0.2em]
Việc tạo ra các chatbot tương tự như mô hình phát triển các ứng dụng và trang web di động và ban đầu bắt đầu với thiết kế. Thiết kế này mô tả sự tương tác của bot và người dùng. Mẫu này cũng bao gồm việc xây dựng bot liên quan đến phân tích đầu vào bằng cách sử dụng một công cụ xử lý ngôn ngữ tự nhiên. Sau các giai đoạn ban đầu, phân tích và bảo trì các bot sau đó được thực hiện.
Việc phát triển Chatbot có thể được thực hiện trên các nền tảng được cung cấp bởi các nhà cung cấp Dịch vụ Nền tảng. Trong số này có IBM Watson, SnatchBot và Oracle Cloud Platform.
Các nghiên cứu gần đây dường như cho thấy mọi người dành nhiều thời gian sử dụng các ứng dụng nhắn tin hơn phương tiện truyền thông xã hội. Do đó, các ứng dụng nhắn tin hiện cung cấp nhiều nền tảng hơn cho các công ty và doanh nghiệp để tiếp cận phần lớn người tiêu dùng. Hiệu quả của chatbot, đặc biệt là các ứng dụng sử dụng AI, lôi kéo và khuyến khích các công ty đầu tư vào các loại dịch vụ này.
Sử dụng trong các ngành công nghiệp khác nhau
Chatbots có một loạt các ứng dụng trong các lĩnh vực khác nhau như kinh doanh, giáo dục, thông tin và giải trí. Được sử dụng lần đầu tiên trong các trò chơi tương tác trực tuyến và nhắn tin tức thời, các bot này hiện đang được phân loại dựa trên việc sử dụng chúng trong giao tiếp, phân tích, thiết kế, du lịch, thể thao, mua sắm, cá nhân, thực phẩm và sức khỏe.
Trong lĩnh vực kinh doanh, nhiều công ty đã sử dụng chatbot để cải thiện dịch vụ của họ và tăng doanh số bán hàng của họ. Từ việc xử lý các đơn đặt hàng trực tuyến đến tiếp thị và dịch vụ khách hàng và hỗ trợ, các bot hỗ trợ và phục vụ nhu cầu của người tiêu dùng bất cứ lúc nào trong ngày, có hoặc không có đại lý trực tiếp. Các hãng hàng không và các công ty thương mại điện tử cũng đã sử dụng chatbot trên trang web của họ để cung cấp thông tin và trả lời các câu hỏi cho khách hàng và khách hàng của họ.
Một số công ty cũng đã đầu tư vào chatbot cho các vấn đề nội bộ, chẳng hạn như trong nguồn nhân lực. Các bot có thể hỗ trợ xử lý và bảo mật tài liệu. Các ngành công nghiệp ngân hàng cũng đang đầu tư vào chatbot thay cho các đại lý trung tâm cuộc gọi để phục vụ khách hàng của họ. Hơn nữa, các tập đoàn đồ chơi đang sử dụng đồ chơi dựa trên chatbot. Những thứ này cho phép trẻ tương tác tốt hơn với đồ chơi, cho phép chúng học tốt hơn.
Nói chung, việc sử dụng chatbot trong các ngành tạo ra doanh thu lớn hơn trong khi tiết kiệm thời gian và tiền bạc. Họ cũng hướng dẫn người tiêu dùng tìm kiếm những gì họ quan tâm, mang lại cho họ trải nghiệm tốt hơn và độc đáo hơn.

\textbf{Sự xuất hiện của trợ lý ảo} \\[0.2em]
Chatbots cũng hoạt động như trợ lý ảo. Một số trong số các bot này là những người nổi tiếng mà bạn có thể đã nghe nói về Amazon Alexa, Google Assistant và Siri của Apple. Những trợ lý ảo này có thể thực hiện cả các nhiệm vụ đơn giản và cơ bản, cho phép người tiêu dùng tập trung vào những thứ quan trọng hơn.
Những bot trợ lý ảo này giúp mọi người theo nhiều cách. Họ giữ cho người dùng của họ thông báo với các tin tức hiện tại và cập nhật thời tiết. Họ nhắc nhở họ về lịch trình của họ và hỗ trợ họ trong cửa hàng tạp hóa và tài chính của họ. Quan trọng hơn, bot cũng hoạt động như một người bạn và người bạn tâm tình. Ở châu Á, có một quốc gia có bot mà hàng triệu người nói chuyện. Vì vậy, thật an toàn khi nói rằng có vô số khả năng với bot.

Thành phần con người
Sự phát triển của công nghệ và sự phát triển của chatbot chắc chắn đã thay đổi và cải thiện cách mọi người giao tiếp. Nó đóng vai trò là cầu nối giữa doanh nghiệp và người tiêu dùng, giúp họ hoàn thành nhiệm vụ và đạt được mục tiêu của mình bất kể họ có thể ở đâu.

Chatbots, vào cuối ngày, được lập trình với các mã và lệnh mà mọi người tạo ra. Công nghệ này không hoàn hảo vì nó vẫn đang phát triển và do đó, dễ bị lỗi. Ngoài ra, việc sử dụng độc hại các chatbot trong quảng cáo và spam đã được báo cáo để lôi kéo mọi người và thu thập thông tin cá nhân.
Bots và các công nghệ trí tuệ nhân tạo khác vẫn được con người theo dõi và duy trì. Khi giao dịch với bot, người tiêu dùng nên cẩn thận và bảo vệ an ninh và quyền riêng tư của họ mọi lúc.

Sự phát triển của công nghệ đã dẫn đến sự phát triển của trí tuệ nhân tạo và chatbot, mở ra cơ hội đáng kể cho các doanh nghiệp và công ty đồng thời mang lại sự tiện lợi cho người tiêu dùng. Tuy nhiên, như dự kiến, có những người cho rằng nó giới hạn các tương tác xã hội thực tế của con người.

Vâng, bất chấp sự tiện lợi mà nó mang lại, tương tác với bot và các công nghệ tương tự khác không thay thế (hoặc gây hại) cho sự tương tác của con người. Rốt cuộc, sự thành công của công nghệ phụ thuộc vào chính những người sử dụng nó và những người đứng đằng sau nó.
Sử dụng các tiện ích nhắn tin chatbot có thể gây hại tới khả năng nhận thức và tương tác của con người dẫn đến kỹ năng giao tiếp không được trau dồi, con người trở nên thô lỗ và hung hăng,
Mọi người đang trở nên lười biếng với thông tin và không quen kiểm tra mọi thứ hoặc không bao giờ dành thời gian suy nghĩ.
